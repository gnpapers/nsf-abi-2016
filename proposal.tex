\documentclass{proposal}
\usepackage{epsfig}

\usepackage{geometry}
\usepackage{marginnote}

% NSF proposal generation template style file.
% based on latex stylefiles  written by Stefan Llewellyn Smith and
% Sarah Gille, with contributions from other collaborators.

\newcommand{\mbl}{\textit{Marine Biological Laboratory}}
\renewcommand{\refname}{\centerline{References cited}}

% this handles hanging indents for publications
\def\rrr#1\\{\par
\medskip\hbox{\vbox{\parindent=2em\hsize=6.12in
\hangindent=4em\hangafter=1#1}}}

\def\baselinestretch{1}

\begin{document}

\begin{center}
{\Large{\bf Project Summary}}\\*[3mm]
{\bf Creation of Integrated Name Resolution Service} \\*[3mm]

Dmitry Mozzherin, Richard Pyle, Yury Roskov \\


\end{center}

Scientific names are the metadata which allowed to aggregate and exchange
biological information for the last 250 years. Such role of scientific names
became even more important with advances in informatics. Biology is
transitioning fast into the realm of ``Big Data''. However connecting
information via scientific names is not trivial, because of many spelling
variants of the same name, evolution of binomial names via creation of new
genus-species epithet combinations, homonyms, etc.

\marginnote{Dima: notes can be leaved like this}

The resolution process of a scientific name consists of three stages. The
first stage is lexical --- it is necessary to figure out if a given spelling of
a name belongs to a specific lexical group of name-strings. If yes -- we are
able to reconcile the given spelling with the previously collected information.
Second stage includes finding of all ``objective synonyms'' of a name, which
traces to all known nomenclatural events the name went through and determining
its ``basionym'' corresponding to original description of the name in the
literature. The third stage consists in determination of ``subjective''
synonyms and determination of currently used name according to a recognized
taxonomic authority.

Global Names Architexture, which went through 2(3?) rounds of funding by NSF
was created to solve first two stages --- Global Names Index does lexcial
analysis and placement of a name-string into a lexical group, and Global Names
Usage Bank traces nomenclatural events of a name to find ``basionym'' and all
subsequent objective synonyms for a lexical group. Global Names does not
have the final third component --- taxonomical resolution which determines
``subjective'' synonyms of a name and determination of currently used name. The
most advanced, state of the art project that addresses taxonomic resolution is
the Catalogue of Life. It is estimated that Catalogue of Life contains
taxonomical information about $\approx85\%$ of all known species.

We are excited to propose an integration of Global Names lexical and
nomenclatural reconciliation of names with taxonomical approach of Catalogue of
Life to create a complete reconciliation/resolution workflow. We plan to
cross-map data of Catalogue of Life with Global Names Usage Bank,
build tools which will significantly speedup the completion of Catalogue of
Life. We want to create a workflow which will allow both Global Names and
Catalogue of Life exchange missing data and provide much richer services to
their customers.

\noindent
{\bf Intellectual Merit}

\ \\

\noindent
{\bf Broader Impacts}

\renewcommand{\thepage} {B--\arabic{page}}

\newpage

% reset page numbering to 1.  This is helpful, since the text can only
% be 15 pages, and reviewers will want to believe we've kept within
% those limits

\pagenumbering{arabic}
\renewcommand{\thepage} {D--\arabic{page}}

\newpage

\centerline{\bf Results from Prior NSF Support}

\noindent
{\bf Previous Award Title}
\textit{award number} \(PI\); dates, \$amount

Research carried out goes here

\ \\

\noindent{\Large \bf PROJECT DESCRIPTION}

\section{Introduction}

Text goes here

\section{A Section}

More text.

\section{Another Section}

More text.  Cite an example~\cite[]{Patterson2016}

\section{Yet Another Section}

\section{Time Line and Management Plan}

\section{Summary:  Significance of proposed work}

\subsection{Intellectual Merit}

\subsection{Broader Impacts}


\newpage
\pagenumbering{arabic}
\renewcommand{\thepage} {E--\arabic{page}}

\bibliography{draft}
\bibliographystyle{jponew}

\newpage
\pagenumbering{arabic}
\renewcommand{\thepage} {G--\arabic{page}}
\noindent{\Large \bf BUDGET JUSTIFICATION}

\end{document}
